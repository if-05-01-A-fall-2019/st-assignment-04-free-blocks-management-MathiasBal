\documentclass[a4paper,11pt]{exam}

\usepackage[german, english]{babel}
\usepackage[utf8]{inputenc}
\usepackage{listings}

\usepackage{multirow}

\pointpoints{\%}{\%}

\begin{document}

\printanswers

\begin{center} \fbox{\fbox{\parbox{5.5in}{\centering
IF.06.01 TINF Operating Systems -- Free Blocks, Quotas -- Exercises.}}}
\end{center}
%\makebox[\textwidth]{Name: \enspace\hrulefill}

\begin{questions}


\question [10]{\bf Free Blocks Management Using a Linked List} Consider a file system managing free blocks by using linked lists. The table below shows the final two blocks storing free blocks. Fill the empty tables below to show the changes which occur in the tables after the following scenarios. Highlight the changes using a color pencil.

\begin{parts}
	\part Five new blocks are allocated
	\part The block 22 is freed
	\part Another 5 blocks are allocated
	\part Another block is allocated
	\part Another three blocks are allocated
	\part Four blocks (23456, 8345345, 56, and 634534) are freed
\end{parts}

\begin{tabular}{|c|c|c|}
\hline
Block \# & 17 & 18 \\ \cline{1-3}
Next Block & 18 & 0 \\ \cline{1-3}
\multirow{5}{*}{} & 4589 & 24353 \\ \cline{2-3}
	& 43546 & 98745 \\ \cline{2-3}
	& 718 & 76345 \\ \cline{2-3}
	& 345 & 9877 \\ \cline{2-3}
	& 23456 & 7345 \\ \cline{2-3}
	& 8345345 & 34535 \\ \cline{2-3}
	& 634534 & 154698 \\ \cline{2-3}
	& 3478 & 967 \\ \cline{2-3}
	& 56 & 8657 \\ \cline{1-3}
\end{tabular}

\begin{tabular}{|c|p{3.25em}|p{3.25em}|}
\hline
Block \# & & \\ \cline{1-3}
Next Block & & \\ \cline{1-3}
\multirow{5}{*}{} & 4589 & 24353 \\ \cline{2-3}
	& 43546 & 98745 \\ \cline{2-3}
	& 718 & 76345 \\ \cline{2-3}
	& 345 & 9877 \\ \cline{2-3}
	&  & 7345 \\ \cline{2-3}
	&  & 34535 \\ \cline{2-3}
	&  & 154698 \\ \cline{2-3}
	&  & 967 \\ \cline{2-3}
	&  & 8657 \\ \cline{1-3}
\end{tabular}
\begin{tabular}{|c|p{3.25em}|p{3.25em}|}
\hline
Block \# & & \\ \cline{1-3}
Next Block & & \\ \cline{1-3}
\multirow{5}{*}{} & 4589 & 24353 \\ \cline{2-3}
	& 43546 & 98745 \\ \cline{2-3}
	& 718 & 76345 \\ \cline{2-3}
	& 345 & 9877 \\ \cline{2-3}
	& 22 & 7345 \\ \cline{2-3}
	&  & 34535 \\ \cline{2-3}
	&  & 154698 \\ \cline{2-3}
	&  & 967 \\ \cline{2-3}
	&  & 8657 \\ \cline{1-3}
\end{tabular}
\begin{tabular}{|c|p{3.25em}|p{3.25em}|}
\hline
Block \# & 17 & 18 \\ \cline{1-3}
Next Block & 18 & 0 \\ \cline{1-3}
\multirow{5}{*}{} & & 24353 \\ \cline{2-3}
	&  & 98745 \\ \cline{2-3}
	&  & 76345 \\ \cline{2-3}
	&  & 9877 \\ \cline{2-3}
	&  & 7345 \\ \cline{2-3}
	&  & 34535 \\ \cline{2-3}
	&  & 154698 \\ \cline{2-3}
	&  & 967 \\ \cline{2-3}
	&  & 8657 \\ \cline{1-3}
\end{tabular}
\begin{tabular}{|c|p{3.25em}|p{3.25em}|}
\hline
Block \# & & 18 \\ \cline{1-3}
Next Block & & 0 \\ \cline{1-3}
\multirow{5}{*}{} & & 24353 \\ \cline{2-3}
	&  & 98745 \\ \cline{2-3}
	&  & 76345 \\ \cline{2-3}
	&  & 9877 \\ \cline{2-3}
	&  & 7345 \\ \cline{2-3}
	&  & 34535 \\ \cline{2-3}
	&  & 154698 \\ \cline{2-3}
	&  & 967 \\ \cline{2-3}
	&  & 8657 \\ \cline{1-3}
\end{tabular}

\begin{tabular}{|c|p{3.25em}|p{3.25em}|}
\hline
Block \# & & 18 \\ \cline{1-3}
Next Block & & 0 \\ \cline{1-3}
\multirow{5}{*}{} & & 24353 \\ \cline{2-3}
	&  & 98745 \\ \cline{2-3}
	&  & 76345 \\ \cline{2-3}
	&  & 9877 \\ \cline{2-3}
	&  & 7345 \\ \cline{2-3}
	&  & 34535 \\ \cline{2-3}
	&  &  \\ \cline{2-3}
	&  &  \\ \cline{2-3}
	&  &  \\ \cline{1-3}
\end{tabular}
\begin{tabular}{|c|p{3.25em}|p{3.25em}|}
\hline
Block \# & 634534 & 18 \\ \cline{1-3}
Next Block & 18 & 0 \\ \cline{1-3}
\multirow{5}{*}{} &  & 24353 \\ \cline{2-3}
	&  & 98745 \\ \cline{2-3}
	&  & 76345 \\ \cline{2-3}
	&  & 9877 \\ \cline{2-3}
	&  & 7345 \\ \cline{2-3}
	&  & 34535 \\ \cline{2-3}
	&  &  \\ 23456 \cline{2-3}
	&  &  \\ 8345345 \cline{2-3}
	&  &  \\ 56 \cline{1-3}
\end{tabular}
\question{\bf Free Blocks Management --- Comparision}
Given the two memory footprint scenarios for Free Blocks Management as presented in class. State the condition under which the linked list approach uses less space than the bitmap approach.


The linked list approach uses less space than the bitmap approach when the bitmap has a lot of 1s (1 means "block allocated"). 
\end{questions}
\end{document}
